\documentclass[a4, 11pt]{article}
\usepackage{amsmath}
\usepackage{amsfonts}
\usepackage{graphicx}


\begin{document}


\title{\vspace{-4cm}Pizza Assortment Optimization}
\author{Kunlei Lian}
\maketitle

Let $\mathcal{S} = \{1, 2, \cdots, 10\}$ denote the set of stores under consideration.
Let $\mathcal{T} = \{A, B, C\}$ indicate all the available types of pizzas to display.
The price and cost of a certain pizza type is dependent on the store location and type, and are represented by $p_{st}$ and $c_{st}$, respectively.
The demand of pizza is related to its display quantity $quantity$:

\begin{align}
	d_{st} = \alpha \cdot quantity^{\beta},
\end{align} 
where both $\alpha$ and $\beta$ are given parameters.

\section{Model 1}
In this problem, we need to decide the number of pizzas to display for each type in each store in order to maximize the total profit across the chain.
To model this problem, we define the following decision variables:

\begin{itemize}
	\item $x_{st}$: a nonnegative integer variable that represents the number of pizzas to display for store $s$ and type $t$.
\end{itemize}

The problem could then be formulated as,

\begin{align}
	\text{max.} &\quad \sum_{s \in \mathcal{S}} \sum_{t \in \mathcal{T}} p_{st} \alpha_{st} x_{st}^{\beta_{st}} \label{m1-obj} \\
	\text{s.t.} &\quad  \sum_{t \in \mathcal{T}} x_{st} \leq 20, \ \forall s \in \mathcal{S} \label{m1-cons1} \\
	&\quad \sum_{s \in \mathcal{S}} \sum_{t \in \mathcal{T}} c_{st} x_{st} \leq 100000 \label{m1-cons2} \\
	&\quad x_{st} \in N^0 = \{0, 1, 2, \cdots\}, \ \forall s \in \mathcal{S}, t \in \mathcal{T} \label{m1-cons3}
\end{align}

In this formulation, the objective function \eqref{m1-obj} tries to maximize the total profit across all stores.
The constraints \eqref{m1-cons1} make sure that there are at most 20 pizzas on display in each store.
Constriants \eqref{m1-cons2} is the budget constraint, making sure the total cost does not exeed the given budget limit.
The last constraints \eqref{m1-cons3} define variable types.


\section{Model 2}
In this problem, the stores are divided into 3 groups $\mathcal{G}$ for each pizza type.
To model this problem, we define the following decision variables:

\begin{itemize}
	\item $x_{st}$: a nonnegative integer variable that represents the number of pizzas to display for store $s$ and type $t$.
	\item $v_{stg}$: a nonnegative variable that inidicates the number of type $t$ pizzas to display for group $g$ in store $s$
	\item $y_{stg}$: a binary variable that equals 1 if store $s$ is assigned to group $g$ for pizza type $t$, 0 otherwise.
	\item $z_{tg}$: a nonnegative integer variable that represents the number of type $t$ pizzas to display for group $g$
\end{itemize}

\begin{align}
	\text{max.} &\quad \sum_{s \in \mathcal{S}} \sum_{t \in \mathcal{T}} p_{st} \alpha_{st} x_{st}^{\beta_{st}} \label{m2-obj} \\
	\text{s.t.} &\quad  \sum_{t \in \mathcal{T}} x_{st} \leq 20, \ \forall s \in \mathcal{S} \label{m2-cons1} \\
	&\quad \sum_{s \in \mathcal{S}} \sum_{t \in \mathcal{T}} c_{st} x_{st} \leq 100000 \label{m2-cons2} \\
	&\quad x_{st} = \sum_{g \in \mathcal{G}} v_{stg}, \, \forall s \in \mathcal{S}, t \in \mathcal{T} \label{m2-cons3} \\
	&\quad \sum_{g \in \mathcal{G}} y_{stg} = 1, \, \forall s \in \mathcal{S}, t \in \mathcal{T} \label{m2-cons4} \\
	&\quad \sum_{s \in \mathcal{S}} y_{stg} \geq 2, \, \forall t \in \mathcal{T}, g \in \mathcal{G} \label{m2-cons5} \\
	&\quad v_{stg} \geq z_{tg} - (1 - y_{stg}) M, \, \forall s \in \mathcal{S}, t \in \mathcal{T}, g \in \mathcal{G} \label{m2-cons6} \\
	&\quad v_{stg} \leq y_{stg} M, \, \forall s \in \mathcal{S}, t \in \mathcal{T}, g \in \mathcal{G} \label{m2-cons7}  \\
	&\quad v_{stg} \geq 0, \, \forall s \in \mathcal{S}, t \in \mathcal{T}, g \in \mathcal{G} \label{m2-cons8} \\
	&\quad v_{stg} \leq z_{tg}, \, \forall s \in \mathcal{S}, t \in \mathcal{T}, g \in \mathcal{G} \label{m2-cons9} \\
	&\quad x_{st}, v_{stg}, z_{tg} \in N^0, \, \forall s \in \mathcal{S}, t \in \mathcal{T}, g \in \mathcal{G} \label{m2-cons10} \\
	&\quad y_{stg} \in \{0, 1\}, \, \forall s \in \mathcal{S}, t \in \mathcal{T}, g \in \mathcal{G} \label{m2-cons11}
\end{align}

In this formulation, the objective function \eqref{m2-obj} aims to maximize the total profit across all stores.
The constraints \eqref{m2-cons1} make sure that there are at most 20 pizzas on display in each store.
Constriants \eqref{m2-cons2} is the budget constraint, making sure the total cost does not exeed the given budget limit.
Constraints \eqref{m2-cons3} derive the number of type $t$ pizzas to display for store $s$.
Constraints \eqref{m2-cons4} ensure that a store can only to assigned to one and only one group for a certain pizza type $t$.
Constraints \eqref{m2-cons5} require that there are at least two stores in every group of type $t$.
Constraints \eqref{m2-cons6} - \eqref{m2-cons9} are the linearization of $v_{stg} = y_{stg} * z_{tg}$, in which $M$ is a large number.
In our case, $M$ could take the value of 20 as we can have at most 20 pizzas in a store.
The last constraints \eqref{m2-cons11} define variable types.
	
\end{document}